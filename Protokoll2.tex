\documentclass[a4paper]{article}
\usepackage{graphicx}
\usepackage[ngerman]{babel}
\usepackage{caption, booktabs}
\usepackage{ltxtable}
\usepackage{float}
\selectlanguage{german}


\begin{document}


    %titelblatt & Diskussion -> Caspar
    %Spielelogik, Schaltplan, Funktionen -> Grill

    
    \begin{figure}[h!] %das [h!] macht das das Foto auch wkl an der richtigen Stelle ist
		\includegraphics[width=\linewidth]{Image/Titelblatt.png}
	\end{figure}

    \newpage
    \tableofcontents
    \newpage

    %\titelblatt??????????????
    \section{Spielelogik}
     In dieser Übung wird das Spiel "Pong" programmiert. Das Spielfeld wird mit acht LEDs simuliert, die Anzahl der LEDs kann allerdings aufgrund der dynamischen programmierung schnell verändert werden, ohne dass sich die Spiellogik bzw. die Funktion ändert. Die Position des Balles wird durch aufleuchten der entsprechenden LED angezeigt. Wobei die LED ganz links den Annahmebereich des linken Spielers entspricht und die LED ganz rechts den Annahmebereich des rechten Spielers.\\
     Das Spiel ist so aufgebaut, dass zwei Spieler gegeneinander Spielen können. Durch betätigen des Tasters S1, wird der Ball vom linken Spieler zurückgeschlagen und durch drücken des Tasters S2 vom rechten Spieler zurückgeschlagen. Wird ein Taster fälschlicherweise betätigt, also wenn sich der Ball nicht im entsprechenden Annahmebereich befindet, hat der Spieler die Runde verloren. Wird der Ball nicht korrekt angenommen, sollen die LEDs fünf Sekunden lang in einer bestimmten Sequenz aufleuchten und der Spieler der verloren hat, bekommt den Ball. Ein neues Spiel wird erst nach einem Aufschlag gestartet.\\
     Außerdem wird die Schwierigkeit erhöht, indem der Ball kontinuierlich beschleunigt. Die Geschwindigkeit des Balles verdoppelt sich alle fünf Sekunden. Bei Spielstart wird die Geschwindigkeit wider auf die Anfangsgeschwindigkeit gesetzt.

     \subsection{Spielparameter}
      Die folgenden Parameter bestimmen zu einen großen Teil den Spielablauf. Falls man einen Wert davon ändern möchte, reicht es diesen in der Datenstruktur typePongConfigurations bzw. typePongState anzupassen, ohne weiteren Code zu ändern.
     \begin{table}[h!]
       \begin{center}
         \begin{tabular}{|l|c|r|}
        \hline 
           \textbf{Variable} & \textbf{Initialwert} & \textbf{Beschreibung}\\
           \hline
           fieldWidth & 23,77 & Bestimmt die größe des Spielfeldes\\
           \hline
           kickBackArea & 2,971 & Bestimmt die größe des Annahmebereichs\\
           \hline
           ballSpeed & 4 & Bestimmt die Ballgeschwindigkeit\\
           \hline
           lampNmbr[  ] & 7 & Anzahl der LEDs\\
           \hline
         \end{tabular}
         \caption{Spielparameter}
         \label{tab:Spielparameter}
       \end{center}
     \end{table}

    \section{Schaltplan}    %Screenshots von MatLab und beschreiben?
    
     \begin{figure}[H]
         \includegraphics[width=\linewidth]{Image/Schaltplan.PNG}
         \caption{Schaltplan}
     \end{figure}

     Wird der Taster S1 betätigt, und somit ein Signal am Eingang I0.0 gelegt, wird der Ball vom linken Spieler angenommen. Analog dazu ist der Taster S2 am Eingang I0.1 für den rechten Spieler.\\
     Die LEDs P1 - 8, an den Ausgängen Q0.0 - 0.7, zeigen die Position des Balles an. Die Anzahl kann beliebig erhöht od verringert werden, vorrausgesetzt die entsprechende Variable, wie in Tabelle \ref{tab:Spielparameter} beschrieben, wird dementsprechend angepasst. 

    \section{Funktion der Programmteile} %Tabellen?
    \begin{table}[h!]
        \begin{center}
          \begin{tabular}{|l|r|}
         \hline 
            \textbf{Teilaufgabe} & \textbf{Funktion gegeben}\\
            \hline
            1. Spieldynamik & Ja\\
            \hline
            2. Visualisierung & Ja\\
            \hline
            3. Manuelle Steuerung & Ja\\
            \hline
            4. Spielphasen & Nein\\
            \hline
            5. Beschleunigung un Animation & Nein\\
            \hline
          \end{tabular}
          \caption{Funktion}
          \label{tab:Funktion}
        \end{center}
      \end{table}

      Die Teilaufgaben 1 bis 3 funktionieren gemäß der Anforderungen. Wobei bei Teilaufgabe 3 bei einer nicht Annahme des Balles, kurzzeitig keine LED aufleuchtet, um besser zu erkennen ob der Ball angenommen wurde oder nicht.\\
      für die Aufgaben 4 und 5 reichte uns die Übungszeit nicht aus um diese zu implementieren.

    \section{Diskussion}

    \newpage
    \listoffigures
    \listoftables
\end{document}